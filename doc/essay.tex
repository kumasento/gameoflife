\documentclass[a4paper,12pt]{article}

\usepackage{xeCJK}
\setCJKmainfont{Songti SC}
\usepackage{fullpage}
\usepackage{url}
\usepackage{setspace}
\usepackage{graphicx}
\usepackage{amsmath}
\usepackage{minted}

\usepackage[backend=biber,style=numeric-comp]{biblatex}
\bibliography{essay.bib}
\usepackage[nottoc]{tocbibind}

\usepackage{indentfirst}
\setlength{\parindent}{2em}

\begin{document}

\title{{\huge 生命游戏的混沌性质}}
\author{
	信息科学技术学院\\
	赵睿哲\\
	{\normalsize \texttt{1200012778}}\\
	{\normalsize \texttt{vincent.zhao@pku.edu.cn}}}
\date{}
\maketitle
\onehalfspacing
\begin{abstract}
细胞自动机和生命游戏是计算理论与数学中非常重要的模型,其中生命游戏是混沌程度最高的一种细胞自动机。本文从细胞自动机的概念出发,给出生命游戏的定义和转换函数和相应的程序实现。之后,首先从枚举统计的角度分析得出了生命游戏的混沌性质与尺寸有关,尺寸越大自然越容易产生混沌的情况。之后,给出Lyapunov指数在生命游戏中的定义,并对不同的尺寸进行求解。最后,给出了与生命游戏的图相关的理论,以及之后从图论角度分析的方向。
\end{abstract}

\section{生命游戏概述}

生命游戏(Game of Life)由英国剑桥大学的数学家康威(John Horton Conway)于1970年提出\footfullcite{gardner1970mathematical},是一个特殊的细胞自动机(Cellular Automaton,简称CA)。从一个随机的初始状态出发,该细胞自动机会随时间一步步演化,并根据初始状态的不同有完全不一样的特征。根据理论与实验可以发现,生命游戏在某些条件下具有显著的混沌特征。本章首先介绍细胞自动机的基本概念,进而分析并实现了生命游戏的基本版本。下一章对生命游戏的混沌性质进行具体分析。

\subsection{细胞自动机}

细胞自动机的概念源自于冯·诺依曼 (John von Neumann) 对自我复制系统 (Self-replication systems) 的研究。冯·诺依曼想要模拟生命的自我复制过程,通过借鉴了乌拉姆 (Stanislaw Ulam) 所设计的晶格模型 (Lattice model),于1940年代设计了细胞自动机的原始样态\footfullcite{schiff2011cellular}。

细胞自动机是是一个离散的系统,其离散性体现在:1) 每个细胞是规则的网格单元,通过明确的边界分隔,网格可以组成一维、二维或者是更高维的欧氏空间;2) 每个细胞具有有限个数的状态,比如“存活”或者“死亡”;3) 细胞自动机的演化过程是时间离散的,可以用离散的时间步进行模拟。

细胞自动机的演化依据确定的规则,演化规则一般与细胞的状态与其邻居的状态共同决定。细胞的邻居定义不固定,一般情况下,一维的细胞自动机取一定半径范围内的细胞作为邻居,二维细胞自动机取上下、左右以及四个对角线方向上的细胞作为邻居 (即摩尔邻居,Moore Neighbourhood,如图\ref{fig:moore})。

\begin{figure}[!ht]
\centering
\includegraphics[width=0.3\textwidth]{images/CA-Moore.png}
\caption{摩尔邻居的样例,图例为二维平面细胞自动机,红色网格为邻居细胞,蓝色网格为目标细胞}
\label{fig:moore}
\end{figure}

下一节具体介绍生命游戏这一种细胞自动机的实现。

\subsection{生命游戏}\label{sec:gol}

生命游戏是一种特殊的细胞自动机,它的定义如下:首先,生命游戏是个二维的细胞自动机,邻居的定义与摩尔邻居相同;其次,细胞状态有“存活”和“死亡”两种,细胞的下一个状态由当前状态和邻居状态共同决定。生命游戏的演化规则如下:

令$S_{i,j}^t \in \{ 0, 1 \}$为时刻$t$时位于坐标$(i,j)$的细胞状态:取值为0,表示“死亡”;或者1,表示“存活”。下一个状态$S_{i,j}^{t+1}$取决于当前状态和邻居的状态之和,令邻居状态值之和为$\overline{S}_{i,j}$,即:
$$
\overline{S}_{i,j} = \sum_{x=i-1}^{i+1} \sum_{y=j-1}^{j+1} S_{x,y}^t 
$$

则转换函数为:
\begin{equation}\label{eq:mapping}
S_{i,j}^{t+1} = f(S_{i,j}^t) = 
\begin{cases}
0 & \texttt{if } S_{i,j}^t = 0 \wedge  \overline{S}_{i,j} \neq 3 \\
  & \texttt{or } S_{i,j}^t = 1 \wedge (\overline{S}_{i,j} \leq 1 \vee \overline{S}_{i,j} \geq 4) \\
1 & \texttt{if } S_{i,j}^t = 0 \wedge \overline{S}_{i,j} = 3 \\
  & \texttt{or } S_{i,j}^t = 1 \wedge 2 \leq \overline{S}_{i,j} \leq 3 \\
\end{cases}
\end{equation}

从生态系统的角度理解,$\overline{S}_{i,j}$的值可以衡量一个细胞的邻居范围内的“拥挤”情况:如果当前细胞的状态是“存活”,并且周围过于拥挤的话,那么该细胞就会“死亡”;同时$\overline{S}_{i,j}$值过小,代表周围“存活”细胞太少,那么当前细胞也会因为缺少“同伴”而“死亡”。

\subsection{生命游戏的实现}

本文使用Python语言实现了一个简单的生命游戏小程序,如代码\ref{lst:gol}所示
\footnote{代码开源于\url{https://github.com/kumasento/gameoflife}}。类Life中保存了当前时刻的每个细胞的状态,在\texttt{state}中。细胞的状态为一个二维数组,每一个元素取值为0或者1,初始化时为随机生成。
\singlespacing
\begin{listing}[!ht]
\inputminted[
	breaklines,
	mathescape,
	linenos,
	numbersep=5pt,
	fontsize=\footnotesize]{py}{codes/life.py}
\label{lst:gol}
\caption{生命游戏的简单Python实现}
\end{listing}
\onehalfspacing

此外,上述代码使用了卷积(行15)进行计算优化。\texttt{w}为卷积权值矩阵,对任意一个细胞,周围邻居的权值均为1,自己的权值为10,这样就可以把\ref{eq:mapping}中的映射函数变为一个简单的转换表。具体来说,该转换表的输入取值从0到18,只有取值为3(对应于\ref{eq:mapping}的第三行条件)或者取值为12、13(对应于\ref{eq:mapping}的第四行条件)才能让状态在下一步转换为“存活”,即1。

\begin{figure}[!ht]
	\centering
	\includegraphics[width=0.4\textwidth]{images/gol.png}
	\label{fig:gol}
	\caption{50 X 50的生命游戏的演化阶段之一}
\end{figure}

生命游戏的展示使用了\texttt{matplotlib}作图函数库来实现,制作一个二维坐标图,如果对应的坐标细胞为存活则染色为黑色,否则为背景色(白色)。系统的演化的动画效果通过不断刷新图像来实现。最后的演示效果图\ref{fig:gol}。

在使用上述代码进行演示时,会发现生命游戏随着初始结构的不同,会有完全不一样的演化结果,可以说体现了一定的混沌性质。下一章对生命游戏的混沌性质进行具体的分析。

\section{生命游戏的性质}

生命游戏的演化结果对初始状态十分敏感,演化的形态多种多样。本章首先给出几种生命游戏常见的形态,之后进行演化性质进行理论推导。

\subsection{驱动力与耗散力}

正如第\ref{sec:gol}节对生命游戏的分析,生命游戏可以看做是对生态系统的某种程度上的模拟。整个系统的演化取决于驱动力和耗散力之间的相互作用。本论文认为生命游戏的驱动力是令细胞从“死亡”到“存活”或者保持“存活”状态的力,而耗散力则是令细胞从“存活”走向“死亡”的力。更具体一些,即为映射\ref{eq:mapping}中的两种结果对应的条件。令下一时刻状态为0的条件都是系统的耗散力,令下一时刻状态为1的条件都是系统的驱动力。

\subsection{演化过程分析}

正如前述,生命游戏有各种不同的演化种类,有的经过简单的几步即可收敛为静止,有的会进行周期性的转变,有的甚至进入混沌状态。

为了简化分析过程,本章以一个$N \times N$的有限网格作为分析对象。在该网格边界的细胞,可以认为它们的值为0。这种做法会导致位于边上的细胞更容易存活,位于角上的细胞更容易死亡。

通过简单的推理可以发现,当$N$的取值比较小时,整个生命游戏系统对应的映射函数也比较简单,应该比较容易收敛到定常解或者进入周期的变化。当$N$取值较大时,在有限的演化步中很可能无法收敛,处于混沌状态。因此本节接下来按照不同的$N$的尺度进行分析,对小尺度的$N$\textbf{定量}分析定常解跟周期解的形态,对大尺度的$N$则进行定性的分析。需要注意的是,即使是小尺度的$N$,初始化状态也有$2^{N^2}$种,当$N=5$时已经有$33554432$种情况。因此对小尺度的分析只针对$2 \leq N \leq 5$。同时,这么多种情况很难使用手工推导和识别,因此使用程序进行分析。

如何使用程序进行分析?首先,该程序的功能是枚举所有的初始状态,并计算其最终的收敛结果。因此最关键的是如何判断生命游戏是否收敛。本论文中使用的策略如下:给定迭代次数\texttt{max\_iter},迭代完成后,从最后一个迭代结果出发判断是否满足从$1$开始的周期数。因此能找到的周期上界是$\lfloor \mathtt{max\_iter} \rfloor$,如果系统的周期数比该数值高或者无法得到周期解,都判断为混沌状态。该算法的时间复杂度为$O(2^{N^2} \times N^2 \times \mathtt{max\_iter})$,即对所有初始状态($2^{N^2}$个)计算迭代$\mathtt{max\_iter}$次的结果,每次迭代的复杂度与卷积计算在一个数量级内。

同时,该程序使用\texttt{matplotlib}库生成定常解和周期解的示例图,方便识别规律和验证。此外,使用\texttt{multiprocessing}库对计算进行并行加速,在16核的Intel服务器上,原先需要几天才能算完的$N=5$的情况,现在只需要5.8个小时即可计算完毕。

\subsubsection{定常解与周期解}

对于小尺度的生命游戏收敛结果的统计如下::
\begin{table}[!ht]
\centering
\begin{tabular}{|l|l|l|l|l|}
	\hline
	周期数 & $N=2$ & $N=3$ & $N=4$ & $N=5$    \\ \hline
	1   & 16    & 510   & 62976 & 30466005 \\ \hline
	2   & 0     & 2     & 2512  & 2606499  \\ \hline
	3   & 0     & 0     & 48    & 0        \\ \hline
	4   & 0     & 0     & 0     & 375176   \\ \hline
	5   & 0     & 0     & 0     & 80       \\ \hline
	8   & 0     & 0     & 0     & 27072    \\ \hline
	12  & 0     & 0     & 0     & 79600    \\ \hline
\end{tabular}
\label{table:stats}
\caption{对小尺度的生命游戏的收敛结果的统计}
\end{table}

通过上述统计结果可以得到的结论如下:

首先,$N=2$与$N=3$非常容易收敛,这是因为可以变化的细胞太少的原因,很容易系统就只剩下处于稳定状态的定常解或者周期解。$N=3$的周期解只有周期为2的情况,并且只有如下一种模式:

\begin{figure}[!ht]
\centering
\includegraphics[width=0.5\textwidth]{images/img32.png}
\label{fig:n3}
\caption{$N=3$时出现的周期2模式}
\end{figure}

其次,对于$N=4$或$N=5$的情况,在周期2的时候也重复了图\ref{fig:n3}的模式。但对于其他的周期数而言,得到的模式就比较难以发现了,比如$N=5$的周期数为12的模式之一(如图\ref{fig:n5})。

\begin{figure}[!ht]
\centering
\includegraphics[width=0.8\textwidth]{images/img512.png}
\label{fig:n5}
\caption{$N=5$时的一种周期数为12的模式}
\end{figure}

总之,随着生命游戏的尺度越来越大,产生的模式越来越复杂。可以看出,从$N=2$开始,周期解的个数占比越来越多,同时周期数也越来越大。可以预计,当$N$取值达到一定规模时,很可能会出现无限的、无法收敛的情况。

\subsubsection{混沌}

当生命游戏的尺度更大的时候,实际上生命游戏的结果已经变为“不可判定的”了,即给定一个初始状态$S_0$和一个可能的演化结果$S'$,没有算法可以直接给出从$S_0$是否可以演化到$S'$。如果上述问题可解,那“停机问题” (halting problem) 也可解,而这是不可能的
\supercite{berlekamp2004winning}。这从一个侧面反映了生命游戏的复杂度。

衡量混沌敏感初条件一般使用Lyaponov指数$\lambda$,即从初始状态到第$n$步的变化的平均值:
\begin{equation}
\lambda(n) = \frac{1}{n}\sum_{i = 0}^{n} ln|f'(x_i)|
\end{equation}

对于混沌的系统而言,系统的变化不会收敛,通常定义的混沌即根据$\lambda > 0$来判定。但是对于生命游戏,或者一般的细胞自动机而言,$f$的形式不好确定:$f$必须要反映系统的状态,但是对于细胞自动机而言,系统的状态由所有细胞同时决定。

因此,本论文采用另外一种策略,使用函数$\Delta$来定义两个生命游戏状态的“距离”。令:
\begin{equation}
\Delta(S,S') = \frac{1}{N} \sum_{x=1}^{N}\sum_{y=1}^{N} |S_{x,y}-S'_{x,y}|
\end{equation}
因此$\Delta$的取值从0到$N$。之所以不使用$N^2$平均值,是因为这样的话变化值小于1,$\lambda$则一定小于0。

则可以归纳得到生命游戏的$\lambda$定义:
\begin{equation}\label{eq:lyapunov}
\lambda(t) = \frac{1}{t} \sum_{i=1}^{t} ln\Delta(S_{i-1},S_i)
\end{equation}

使用程序对从$N=2$开始到$N=8192$结束的各种生命游戏系统进行Lyapunov指数进行计算,指定迭代次数足够大 (大于100000),得到统计结果如下:

\singlespacing
\begin{listing}[!ht]
	\inputminted[
	breaklines,
	mathescape,
	linenos,
	numbersep=5pt,
	fontsize=\footnotesize]{py}{codes/lambda.py}
	\label{lst:lyapunov}
	\caption{生命游戏的Lyapunov指数计算}
\end{listing}
\onehalfspacing

\begin{table}[!ht]
\centering
\begin{tabular}{|l|l|}
	\hline
	$N$  & Lyapunov指数 \\ \hline
	2    & -13.694151 \\ \hline
	16   & -3.934968  \\ \hline
	64   & 1.805572   \\ \hline
	256  & 3.289903   \\ \hline
	1024 & 4.707479   \\ \hline
	8196 & 6.786478   \\ \hline
\end{tabular}
\label{table:lyapunov}
\caption{生命游戏的Lyapunov指数}
\end{table}

可以看出,随着生命游戏的尺度不断变大,则Lyapunov指数为正,并且不断增加。可以认为生命游戏具有明显的混沌特征。

\subsection{混沌分析的另一种角度 —— 图论}

之前的章节把生命游戏看做是简单的映射,但是考虑到生命游戏本身具有明显的离散性质,每一个状态是有限可数的,因此可以使用有向图来描述生命游戏之间的状态转换。

给定生命游戏的尺度$N$,那么可以枚举得到$2^{N^2}$种状态,从0开始给每一个状态进行编号,每个编号的二进制形式的每一位都代表一个细胞的“存活”(值为1)和“死亡”(值为0)状态。令每个状态为图中的一点,那么该图有如下性质:首先,每个图中的结点只有一条出边,即从一个状态经一步转换只能到一个唯一的状态;其次,每个结点可以有多条入边;最后,结点的出边可以指向自己,即形成一个结点的环。在该图中的环即为状态转换的周期,周期数为环中的结点个数。

上述图的性质中,最重要的是每个结点只能有一条出边,因此很容易证明,如果从两个不同结点出发沿边的方向进行状态转换,那么当两条移动路径在同一时刻达到同一个结点的时候,之后会拥有一样的状态转移路径。也即,假如状态转移路径中访问了两次相同的节点,那在这两次之间一定会形成周期,即环。那么,因为状态的个数是有限的,图中只有有限个结点,因此对于生命游戏,当迭代次数大于生命游戏的结点个数的时候,一定会进入周期状态。

基于该结论可以如何分析出混沌性质目前还不清楚,不过可从如下两个角度对生命游戏的混沌性质进行探究:第一,当状态转移次数足够大的时候,转移结果是否能确定?第二,在该图中两个相似的节点,最终的状态转移路径相差多少?相似性的定义与$\Delta$一致。

总之,在图论的基础上,整个生命游戏的状态转移更清楚了:生命游戏本质上是离散的状态转移,很难像Logistic映射一样给出十分清晰的结论,因此使用从图论的角度分析或许能得到更形象的结果。

\section{结论}

本论文通过使用计算机程序的方法对生命游戏的状态演化进行了模拟,最终的结论是生命游戏的确具有混沌的特征:周期数和Lyapunov指数都证明了这一点。但是本文依然有很大的局限性:首先,对于$N>6$的情况下使用枚举的办法几乎无法实现,需要对模拟的算法进行更多优化;其次,生命游戏更深层次的规律单纯使用程序的手段难以发现,必须得从数学的角度进行证明,而本文并未能完成这个角度的工作。未来可以尝试从图论的角度对生命游戏做更细致、更形象的分析。

\newpage
\printbibliography

\end{document}